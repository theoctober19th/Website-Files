\documentclass [12pt, a4paper]{article}
\setlength{\parindent}{0pt}



\usepackage[left=2.54cm, right=2.54cm, top=2.54cm, bottom=2.54cm]{geometry}

\usepackage{graphicx}
\graphicspath{ {images/} }

\usepackage{float}

\title{Assignment: Social and Professional Issues in Information Technology}
\author{\textbf{Bikalpa Dhakal, 15395}}
\date{14 AUG 2019}

\begin{document}

\begin{titlepage}
	\maketitle
	\vfill
	\underline{The source for this LaTeX docuement can be found at : }\\
	\textbf{bkalpa.com.np/docs/academia/spit/assignment.tex}
\end{titlepage}

% HERE STARTS THE ASSIGNMENT PORTION

\begin{center}
	\large
		\textbf{ \underline{2019 Fall} }
	\normalsize
\end{center}

\large
\textbf{1. a) Describe the contribution of Edward Dijkstra in the field of computing.}

\normalsize
	
Edgar Dijkstra was a Dutch system scientist, programmer, software engineer, science essayist and one of the major pioneers in computing science. His major works and contributions are listed as follows:

\begin{itemize}

	\item Prior to him, computer programming was considered more of an art rather than a scientific discipline. He played a significant role to establish computer programming as one of a \textbf{scientific discipline.}
	\item He was the first person to introduce the concept of \textbf{structured programming} in programming arena. Many programming languages like C, C++ etc. were developed using this concept.
	\item He was the first person to state and solve the \textbf{mutual exclusion problem} in concurrent programming, with his influential paper titled 'Solution of a Problem in Concurrent Programming Control'. He proposed the semaphore mechanism for mutual exclusion algorithm, which was probably the fist published concurrent algorithm and which introduced new area of algorithmic research. He also identified the \textbf{deadlock} problem and proposed the famous \textbf{Banker's algorithm} that prevents deadlock. He is also accredited for the formulation of the \textbf{Dining Philosophers Problem} and \textbf{the Sleeping Barber problem} frequently used in literature of concurrent programming.
	\item He invented an algorithm to find the shortest path in a graph between any two given nodes, which is now called \textbf{Dijkstra's algorithm}. This algorithm is used in routing protocols like OSPF and IS-IS. He also independently re-discovered the algorithm known as Prim's \textbf{minimal spanning tree algorithm}.
	\item Djikstra developed \textbf{smoothsort}, a comparison-based sorting algorithm and a variation of heapsort.
	\item He wrote jointly with Jaap Zonneveld the first \textbf{ALGOL 60 compiler}.This was the fist compiler to support \textbf{recursion}.
	\item He helped standardize the computer programming by publishing many papers and articles. One of them was the famous \textbf{'GoTo Statement Considered Harmful'}, where he discussed why goto statements in computer programming should be avoided.
	\item He and his colleagues in Eindhoven designed and implemented \textbf{THE Operating System} that introduced the idea of using layers in operating system to control complexity.
	\item He was one of the very early pioneers of the research on principles of distributed computing. He jointly with Carel S. Scholten proposed the \textbf{Dijkstra-Scholten algorithm} for detecting termination in distributed systems.
	
\end{itemize}

\large \textbf{1. b) How has society and technology affected each other? Discuss.}

\normalsize
Society can be defined as a population that occupies the same territory that is subject to the same political authority and that participates in a common culture, whereas, the term technology refers to the collection of techniques, skills, methods and processes used in production of goods or services or in the accomplishment of objective, such as scientific investigation. These two terms are largely interrelated together ever since the concept of society was developed during ancient civilizations. These relationships can be discussed briefly under following fields.

\begin{itemize}
\item \textbf{Economic Aspect} : Economy is one of the major foundations for the development of the society. Technology has made industrialization possible, and day by day the cost of production is decreasing gradually by use of newer and efficient technologies. This has strengthened the economy of the society and has upgraded the quality of life of the people. The use of technology have exposed more and more people to employment opportunities, and created many job opportunities in technical field. Technology has made possible electronic payment and electronic commerce system that has made both sellers and buyers easy in selling/buying products. In the other hand, the economy of society directly influences the pace of development of the technology. A society that is economically backwards cannot invest much money in research and production of advanced technologies whereas a society that is economically stable can.

\item \textbf{Socio-Cultural Aspect}  : Technology has drastically raised the awareness level of people in the society. It has made easier for people to access trustworthy and reliable source of information from all around the world. People are able to share and showcase their traditions and cultures via various social media. The increase in social awareness has drastically reduced the number of crimes in the society. Use of advanced technology in field of education has ensured participation of even more people towards quality education. In other hand, the cultures and norms of a society dictates what sort of advancement is acceptable and what is not.

%\begin{figure}[h!] 
%\centering
%	\includegraphics[width=0.5\linewidth]{tech-and-society}
%\end{figure}

\item \textbf{Political Aspect} : Technology has made possible the concept of e-governance. The use of technology in various public sectors have made people easier to receive services provided by the government. It has been easier for political bodies to reach people and put forward their agendas more efficiently by the use of technology. In other hand, the political environment of the society has direct influence on the development of technology, because it is the politics that governs the laws and acts regarding the use of technology.

\item \textbf{Environmental Aspect} : Technology has raised the awareness of people to make effort on the conservation of the environment. Better waste management techniques, pollution control techniques, mass production techniques etc. have been developed, thanks to the advancements in technology. In other hand, technical advancements require a lot of natural resources and these are not possible in an environment where the resources are depleted and scarce.

\end{itemize}

\large
\textbf{2. a) Case: Facebook-related crimes on Rise: Police} \\
\textbf{...} \\
\normalsize
\textbf{a) Taking the reference of the given case, discuss the current status of issues related to IT in Nepalese society. Relate ethical issues with the scenario.} \\



\normalsize
The given case study accurately portrays the problems prevalent in Nepalese society due to the widespread use of social media like Facebook in Nepal. The cases discussed like phishing, the use of fake Facebook accounts, email thieving, data hacking, blackmailing and frauds are one of the most common cyber crime activities in Nepal.\\
\par
Nepalese people become victim of such fraud attacks because of their innocence and lack of awareness. The notion of cyber crime and awareness have not yet reached to the public completely, and there are many cases in which people commit crimes unintentionally too. Whereas in the cases discussed in the question, such fraudsters have a large network to conduct illegal activities, and even law enforcement bodies have hard time tracking and arresting them all.\\
\par
Adding to the given cases, let us discuss about an event that took place recently in Nepal. Recently (11 Aug), Nepal Police arrested three people from Janakpurdham sub-metropolitan on charges related to illegal exposition of the exam question paper of MBS first semester. These people illegally leaked the question paper by sharing them through Facebook Messenger. Yet another example can be the recent arrest of people on charges of 'Honey Trapping'. This new form of crime first lures rich industrialists and businessmen using a woman, who sexually seduces them and convinces them to go to a hotel to have sex. Later, armed criminals raid the same hotel room, make videos of them, and demand money by blackmailing them. The criminals threat them saying they would make the video public in the internet, and the victims comply to pay the money fearing the public shame. Recently Nepal Police have arrested three people on charges of conducting criminal activities with use of Facebook Page. These people who were operating from India lured people to send them money to buy gadgets and gifts at very price, via a facebook page disguising themselves as an online store. \\
\par
In my opinion, the root cause of these activities is the lack of strict acts and regulations for those who are involved in such criminal activities. As it is itself mentioned in the paragraph given, Nepal Police are currently handling all of these crimes under Electronic Transaction Act 2006, which has long been irrelevant and outdated. It isn't specific towards various forms of cyber crimes that occur in Nepal, so often times the police is confused as to which charge the criminal should be arrested for. It is almost undesputable that we need a comprehensive cyber law to be formulated soon. \\
\par
Many a times, the ethical awareness can be a leading measure against such crimes. People in Nepal have gone poor schooling in terms of ethics, thanks to the traditional and conservative society. If people can decide what is ethical to do and what is not, even in the smallest means, the crimes similar to these would certainly be reduced. Relating to the case given, it is clear to all person that deceiving others is unethical. Such ethical judgement can easily be interrelated to the crimes.\\

\textbf{b) What should be the role of professionals in dealing with the aforementioned condition?} \\
In my opinion, the following are the roles of professionals in dealing with these conditions:
\begin{itemize}
	\item Persuade the governmental bodies and higher authorities to formulate and pass bill in order to bring new act related to cyber law in Nepal, both individually as well as via various professional groups like unions.
	\item Co-operate with Nepal Police to help them build technical infrastructure as well as human resource for investigating cyber crimes.
	\item Share awareness to the common people to refrain from criminal activities that are considered illegal as well as unethical in the field of cyberspace.
	\item Report any suspicions and violations of the laws related to cyberspace immediately to the Nepal Police.
	\item Learning about new technologies and vulnerabilities and implementing them in places the professional work, so that unauthorized breaching would not be so easy.
\end{itemize}


\large
\textbf{3. Case Study}\\
\normalsize
\textbf{3. a) Do producers of widely used reference works have an ethical responsibility to report the substance of their field accurately, or a social responsibility to remove potentially offensive words from the language?}\\
In my opinion, they have ethical responsibility to report the substance of their field accurately. There are plenty of words in the English dictionary which have their synonyms but their meanings cannot be exactly be conveyed by any other words than themselves. What words a writer writes in her article is her choice. The move made by Microsoft to remove those words from the thesaurus is not a good decision from software engineering point of view. No word is intrinsically offensive or negative. It depends upon the context where it is used. In today's world where almost more than a half writers and authors use Microsoft Word to write and format their books and documents, the limitations in the thesaurus will only decrease the lexical beauty and diversity of those works. Eventually, other words will gradually become extinct from the literature. As an example, 'Only an idiot will do it' is not the same as 'Only a fool will do it', or 'Only a trick will do it'. \\

\textbf{3. b) Should software engineers concern themselves with ethical issues any more than other individuals in society?}\\
In my opinion there are two aspects to this argument. In one hand, software engineers are also the people living in the society. So they are no more special than anyone else living in the society. The fundamental theories of ethics applies equally to all individuals in the society. Software engineers should not alter the contents that fall out of their application area while they develop the software solutions. In other hand, there are specific ethical issues that are more relevant from the point of view of a software engineer rather than a common individual in the society. For example, the requirement that a software engineer build a software complying to the security requirements and also not leave backdoor access is a common ethical requirement. This doesn't apply to a common person living in the society.\\

\textbf{3. c) Do computers and computer related technology really raise new ethical problems or do they just present old issues in a new form?}\\
The majority of the ethical problems that arise by the introduction of the computers and technology are just the old issues presented in new form. For example, no person in a company should disclose confidential information about the company itself to the general public. In form of technology, this ethical responsibility can be stated as 'no personnel in a company should disclose the confidential information regarding the company by any means, be it to the press, social media, email, internet or in written form.' This is just the reformulation of the old ethical problem. The problem of data breaching and illegal access is quite similar to the stealing of the files and spying in the old days, the difference is that those activities used to take place physically in old days, while they happen digitally in modern days.\\

\large
\textbf{4. a) "Software risk is considered sensitive in case of life critical systems as the developer is the only person to blame if an accident occurs". Do you agree or disagree? Justify your opinion.} \\
\normalsize
I strongly agree to this statement. In fact, we, the software engineers have to build softwares that are supposed to be used in a field that is completely unknown to us. Many a times, we are developing software solutions that can cause fatal accidents if only a slightest mistake is made. For instance, it is the software engineers that design the autopilot software that are used in commercial aircrafts. It is us who make software for the machines that are used to perform automated operations and surgery. Softwares that are used in self-driving vehicles, or those in missile detection system, etc. are other examples of the life critical systems. In fact, only a slightest error in the software for those areas can bring in a large catastrophe and put human lives in risk. If such things happen, the professionals in that field will obviously blame the programmer for this. 

Here are some of the very popular fatal accidents, that were caused by software errors.\\
\begin{itemize}
	\item The Therac-25 radiation machine killed at least six people due to overdose of radiation, all due to the fault in the software installed within the machine.
	\item In February 1991, an American Patriot Missile battery in Saudi Arabia failed to track and intercept an incoming Iraqi Scud missile due to the errors in the calculation of time in the software, and killed 28 soldiers and injured 100 other people.
	\item In 1997, Korean Air Flight 801 crashed just three miles short of the runway in Guam due to a fault with the Ground Proximity Warning System, killing 228 people.
	\item in June 1994, a Royal Air Force Chinook helicopter crashed into the Mull of Kintyre, killing 29. Investigation later revealed that it was caused due to software bug in the aircraft's engine control computer.
\end{itemize}

These sort of accidents happen rarely, but they can't be said impossible. However rare they may occur, they bring great destruction to the human lives once they occur. As the professionals in the other fields totally rely on the programmers and developers for the proper functioning of the software, software developers should be extremely wary of such bugs. After an accident happens, the person that is legally obliged to answer for the accident is the software developer herself. It is without doubt we can say that software risk is considered sensitive in case of life critical systems.\\

\large
\textbf{4. b) Is it possible to enforce censorship in cyberspace? Justify your opinion with relevant example.}\\
\normalsize
\par
In my humble opionion, it is impossible to enforce a complete censorship in cyberspace in a way that is effective.\\
\par
The internet is a very large network. It is actually a million times larger than we actually think it is. The world wide web, or the WWW, that we use and know of, accounts of only about 10\% of the internet. More than 90\% of the internet consists of the deep web, which are not accessible through search engines like Google. The internet has long exapanded from a mere international network for news, communications and mail, to complicated activities like cryptocurrency, drug smuggling, human trafficking and so on. Even the effort of the entire world to control this network has proved to be unfruitful, and day by day, the exposure of web has been increasing. In the internet and technological world today, people eventually find a way out to bypass censorship. No one actually owns the internet. So, it is quite an impossible task to pose a censorship.\\
\par
Talking about the examples, let us take the example of the recent ban posed by Nepal Government over the porn websites. The home ministry of Nepal decided to ban the porn websites within the Nepal and thus asked various Internet Service Providers to ban those sites for their users. The ban was soon imposed and many people thought that watching porn in Nepal have actually came to an end. But only a few days after, people eventually found out that using a VPN service effectively bypassed the ban imposed by the government. Any people can use a VPN service to change her location in the virtual cyberspace. Let alone VPN, the ban could easily be bypassed by mere use of a DNS from DNS providers like Cloudfare. The Nepal Government cannot refrain its citizens from using secondary DNS service, nor can it say its citizens not to use VPN. Actually, almost all of financial institutions use VPN for their day to day transactions. Also, from ISP's perspective, there is no way that an ISP can know whether a user is using VPN or not. So, I strongly believe that in the era of technology, censoring something off completely is quite not possible.\\

\large
\textbf{5. a) IPR (Intellectual Property Rights) have become a serious concern with the advancement in computing technology. What are some of the intellectual properties? Describe.}\\
\normalsize
Intellectual properties are those properties that can be protected under federal law, including copyrightable works, ideas, discoveries and inventions. Intellectual Property Rights, or IPR in short, are a collection of rights given to the common people to ensure that no other person could misuse the intellectual property synthesized or gathered by that person.

In law, particularly in common law jurisdictions, intellectual property or IPP refers to a legal entitlement which sometimes attaches to the expressed form of an idea, or to some other intangible subject matter. The term intellectual property reflects the idea that the subject matter of IP is the product of the mind or the intellect, and that once established, such entitlements are generally treated as equivalent to tangible property, and may be enforced as such by the courts.

The following are the most well known forms of intellectual properties.

\begin{itemize}
	\item \textbf{Copyright}: Copyrights protect the expression of ideas. Artistic works are generally considered to be expressions of ideas – books, paintings, songs, movies, and computer programs are examples. Copyright will not protect the process through which a particular work was created or the use of information within it (instructions, etc.). Cookbooks are often used to illustrate the difference between the expression of an idea and the idea itself. Cookbooks cannot be reproduced without permission because they are an expression of ideas (the recipes). However, people can still follow the recipes in the cookbook because they are replicating the ideas contained in the literary work. If the recipes were protected by a patent, users would need permission to follow them, since patents protect particular ideas from being used without authorization.
	\item \textbf{Patent}: A patent is a right, granted by the government, to exclude others from making, using, or selling your invention. Patents protect inventions such as new processes, machines, or chemicals. The central idea is that patents protect ideas, not just expressions of them. The main effect of patents is to give their holders the right to challenge any use of the invention by a third party. He thereby gave a temporary monopoly of exploitation which can be understood as a financial incentive for inventive industrial activities.
	\item \textbf{Trademark}: A trademark (also written trade mark or trade-mark) is a type of intellectual property consisting of a recognizable sign, design, or expression which identifies products or services of a particular source from those of others, although trademarks used to identify services are usually called service marks. The trademark owner can be an individual, business organization, or any legal entity. A trademark may be located on a package, a label, a voucher, or on the product itself. For the sake of corporate identity, trademarks are often displayed on company buildings. It is legally recognized as a type of intellectual property.
	\item \textbf{Trade Secret} :A trade secret is a type of intellectual property in the form of a formula, practice, process, design, instrument, pattern, commercial method, or compilation of information not generally known or reasonably ascertainable by others by which a business can obtain an economic advantage over competitors or customers. In some jurisdictions, such secrets are referred to as confidential information.\\
\end{itemize}

\large
\textbf{5. b) What do you mean by digital forensics? Explain different cyber crimes. How would you prevent your computer from crime? Discuss.}\\
\normalsize
Digital forensics (sometimes known as digital forensic science) is a branch of forensic science encompassing the recovery and investigation of material found in digital devices, often in relation to computer crime.\\
\par
Cybercrime is any criminal activity that involves a computer, networked device or a network. Cybercriminals may use computer technology to access personal information, business trade secrets or use the internet for exploitative or malicious purposes. Criminals can also use computers for communication and document or data storage. Criminals who perform these illegal activities are often referred to as hackers.\\
\par
Some common cyber crimes are described as below:
\begin{itemize}
	\item \textbf{Fraud}: Fraud is a general term used to describe a cybercrime that intends to deceive a person in order to gain important data or information. Fraud can be done by altering, destroying, stealing, or suppressing any information to secure unlawful or unfair gain.
	\item \textbf{Hacking}: Hacking involves the partial or complete acquisition of certain functions within a system, network, or website. It also aims to access to important data and information, breaching privacy. Most “hackers” attack corporate and government accounts. There are different types of hacking methods and procedures.
	\item \textbf{Identity Theft}: Identify theft is a specific form of fraud in which cybercriminals steal personal data, including passwords, data about the bank account, credit cards, debit cards, social security, and other sensitive information. Through identity theft, criminals can steal money. According to the U.S. Bureau of Justice Statistics (BJS), more than 1.1 million Americans are victimized by identity theft.
	\item \textbf{Scamming}: Scam happens in a variety of forms. In cyberspace, scamming can be done by offering computer repair, network troubleshooting, and IT support services, forcing users to shell out hundreds of money for cyber problems that do not even exist. Any illegal plans to make money falls to scamming.
	\item \textbf{Computer Viruses}: Most criminals take advantage of viruses to gain unauthorized access to systems and steal important data. Mostly, highly-skilled programs send viruses, malware, and Trojan, among others to infect and destroy computers, networks, and systems. Viruses can spread through removable devices and the internet.
	\item \textbf{Ransomware}: Ransomware is one of the most destructive malware-based attacks. It enters your computer network and encrypts files and information through public-key encryption. In 2016, over 638 million computer networks are affected by ransomware. In 2017, over \$5 billion is lost due to global ransomware.
	\item \textbf{DDoS Attack}: DDoS or the Distributed Denial of Service attack is one of the most popular methods of hacking. It temporarily or completely interrupts servers and networks that are successfully running. When the system is offline, they compromise certain functions to make the website unavailable for users. The main goal is for users to pay attention to the DDoS attack, giving hackers the chance to hack the system.
	\item \textbf{Botnets}: Botnets are controlled by remote attackers called “bot herders” in order to attack computers by sending spams or malware. They usually attack businesses and governments as botnets specifically attack the information technology infrastructure. There are botnet removal tools available on the web to detect and block botnets from entering your system.
	\item \textbf{Spamming}: Spamming uses electronic messaging systems, most commonly emails in sending messages that host malware, fake links of websites, and other malicious programs. Email spamming is very popular. Unsolicited bulk messages from unfamiliar organizations, companies, and groups are sent to large numbers of users. It offers deals, promos, and other attractive components to deceive users.
	\item \textbf{Phishing}: Phishers act like a legitimate company or organization. They use “email spoofing” to extract confidential information such as credit card numbers, social security number, passwords, etc. They send out thousands of phishing emails carrying links to fake websites. Users will believe these are legitimate, thus entering their personal information.
	\item \textbf{Social Engineering}: Social engineering is a method in which cybercriminals make a direct contact with you through phone calls, emails, or even in person. Basically, they will also act like a legitimate company as well. They will befriend you to earn your trust until you will provide your important information and personal data.
\item \textbf{Malvertising}: Malvertising is the method of filling websites with advertisements carrying malicious codes. Users will click these advertisements, thinking they are legitimate. Once they click these ads, they will be redirected to fake websites or a file carrying viruses and malware will automatically be downloaded.

\item \textbf{Cyberstalking}: Cyberstalking involves following a person online anonymously. The stalker will virtually follow the victim, including his or her activities. Most of the victims of cyberstalking are women and children being followed by men and pedophiles.

\item \textbf{Software Piracy}: The internet is filled with torrents and other programs that illegally duplicate original content, including songs, books, movies, albums, and software. This is a crime as it translates to copyright infringement. Due to software piracy, companies and developers encounter huge cut down in their income because their products are illegally reproduced.

\item \textbf{Child Pornography}: Porn content is very accessible now because of the internet. Most countries have laws that penalize child pornography. Basically, this cybercrime involves the exploitation of children in the porn industry. Child pornography is a \$3-billion-a-year industry. Unfortunately, over 10,000 internet locations provide access to child porn.

\item \textbf{Cyberbullying}: Cyberbullying is one of the most rampant crimes committed in the virtual world. It is a form of bullying carried over to the internet. On the other hand, global leaders are aware of this crime and pass laws and acts that prohibit the proliferation of cyberbullying.

\end{itemize}

Cyber crimes are inevitable. However, with proper measures of digital awareness and security, we can protect ourselves to an large extent. The following are some measures to be taken to prevent our computers from cyber crimes:

\begin{itemize}

\item \textbf{Use a full-service internet security suite.} For instance, Norton Security provides real-time protection against existing and emerging malware including ransomware and viruses, and helps protect your private and financial information when you go online.

\item \textbf{Use strong passwords.} Don’t repeat your passwords on different sites, and change your passwords regularly. Make them complex. That means using a combination of at least 10 letters, numbers, and symbols. A password management application can help you to keep your passwords locked down.

\item \textbf{Keep your software updated.} This is especially important with your operating systems and internet security software. Cybercriminals frequently use known exploits, or flaws, in your software to gain access to your system. Patching those exploits and flaws can make it less likely that you’ll become a cybercrime target.

\item \textbf{Manage your social media settings.} Keep your personal and private information locked down. Social engineering cybercriminals can often get your personal information with just a few data points, so the less you share publicly, the better. For instance, if you post your pet’s name or reveal your mother’s maiden name, you might expose the answers to two common security questions.

\item \textbf{Strengthen your home network.} It’s a good idea to start with a strong encryption password as well as a virtual private network. A VPN will encrypt all traffic leaving your devices until it arrives at its destination. If cybercriminals do manage to hack your communication line, they won’t intercept anything but encrypted data. It’s a good idea to use a VPN whenever you a public Wi-Fi network, whether it’s in a library, café, hotel, or airport.

\item \textbf{Talk to your children about the internet.} You can teach your kids about acceptable use of the internet without shutting down communication channels. Make sure they know that they can come to you if they’re experiencing any kind of online harassment, stalking, or bullying.

\item \textbf{Keep up to date on major security breaches.} If you do business with a merchant or have an account on a website that’s been impacted by a security breach, find out what information the hackers accessed and change your password immediately.

\item \textbf{Take measures to help protect yourself against identity theft.} Identity theft occurs when someone wrongfully obtains your personal data in a way that involves fraud or deception, typically for economic gain. How? You might be tricked into giving personal information over the internet, for instance, or a thief might steal your mail to access account information. That’s why it’s important to guard your personal data. A VPN — short for virtual private network — can also help to protect the data you send and receive online, especially when accessing the internet on public Wi-Fi.

\item \textbf{Know that identity theft can happen anywhere.} It’s smart to know how to protect your identity even when traveling. There are a lot of things you can do to help keep criminals from getting your private information on the road. These include keeping your travel plans off social media and being using a VPN when accessing the internet over your hotel’s Wi-Fi network.

\item \textbf{Keep an eye on the kids.} Just like you’ll want to talk to your kids about the internet, you’ll also want to help protect them against identity theft. Identity thieves often target children because their Social Security number and credit histories frequently represent a clean slate. You can help guard against identity theft by being careful when sharing your child’s personal information. It’s also smart to know what to look for that might suggest your child’s identity has been compromised.

\item \textbf{Know what to do if you become a victim.} If you believe that you’ve become a victim of a cybercrime, you need to alert the local police and, in some cases, the FBI and the Federal Trade Commission. This is important even if the crime seems minor. Your report may assist authorities in their investigations or may help to thwart criminals from taking advantage of other people in the future. If you think cybercriminals have stolen your identity. These are among the steps you should consider.

\begin{itemize}
	\item Contact the companies and banks where you know fraud occurred.
	\item Place fraud alerts and get your credit reports.
	\item Report identity theft to the FTC.
\end{itemize}

\end{itemize}

\large
\textbf{6. a) Some people think that technology has made life easier, others claim the opposite. What is your opinion on this issue?}\\
\normalsize
I'm on the positive side of this opinion. There may be some exceptions where the technology have made the day to day activities even more difficult, but apart from those rare exceptions, anyone can say without a doubt that the technlogy has made our life easier.\\
\par
Here are some of the ways in which human lives have been eased by technology:

\begin{itemize}

\item Technology has made communication easier. Earlier on, you had to send write a letter and so on. The recipient had to wait for days and even months. Nowadays, people normally use text messages and emails to pass important messages to colleagues and relatives. If you want to take photographs, you can take a digital camera and upload them directly to your PC. Once you are done, you can attach a copy to an email and send it off to the recipient. This way, you can actually share experiences with friends and relatives.

\item Technology has made advertising easier. There are many ways through which you can advertise your business. You can use social media and so on. You just need to create a website for your business and then create a page on the social media platforms and encourage your friends and relatives to like and share the page. This can go a long way in increasing the conversion rates and eventually more sales in your business.

\item Nowadays, you can easily locate places by using Google Earth Maps and so on. You just need to install this app on your smartphone or tablet and use it to locate places. This can be very helpful especially when you are going for adventures and so on. In addition, when you are going to new places, you can always locate your next destination so that you avoid getting lost on the way.

\item Most people have ever misplaced their wallets, car keys, mobile phone and so on. This can actually frustrate you. You will realize that once you have misplaced your car keys, you cannot access your car. However, you can actually replace your car keys and save enough time trying to find your lost key, faster than ever.\\
Instead, you can also use app like TrackR that can help you to stay connected to the things that you value most. TrackR has an app for both iPhone and your smartphone. This can actually help you locate your car keys in case you misplace them.

\item Nowadays, you can get any information on the internet. If you want to know more about somebody or something, you just need to visit the search engines and search for the content that you want. Take for example the information about people's money. We would not have known about the richest footballers in the world if there was no internet. This has made learning easier. If you have assignments, you can get the information that you need on the internet.

\item Earlier on, people had to manually file the important data of their customers. Information was stored in files and so on. This actually made retrieving of data more difficult. It used to be time consuming and so on. Nowadays, you can actually use a computer to save data and retrieve it. In just once click, you can instantly find the name of a customer, the birth date, address and so on.\\
This actually saves time and so on. In addition, this ensures that the data remains safe from damage and so on.Those are some of the ways through which technology has made our lives easier. Sharing of experiences has been made easier. With social media platforms, you can actually share pictures with your loved ones and friends. Nowadays, you can also buy items online unlike before.\\
In addition, computer technology has brought a lot of flexibility. Nowadays, you do not have to work in an office. You can actually work from home. Truly, technology has made life easier. It is truly incredible.

\end{itemize}

\large
\textbf{ 6. b) Cyber Terrorism has been a major issue in today's context. What do you think how to deal with this issue? Put your views.}\\
\normalsize
Cyberterrorism is the use of the Internet to conduct violent acts that result in, or threaten, loss of life or significant bodily harm, in order to achieve political or ideological gains through threat or intimidation. It is also sometimes considered an act of Internet terrorism where terrorist activities, including acts of deliberate, large-scale disruption of computer networks, especially of personal computers attached to the Internet by means of tools such as computer viruses, computer worms, phishing, and other malicious software and hardware methods and programming scripts.\\
\par
Today, where every people in the world have easy and unrestricted access to the internet, the crimes like cyber terrorism has increased significantly.

Over recent years vulnerabilities in softwares and new technologies have proved that security is not often at the forefront of priority during it’s development. An example being, The Internet of Things (IoT) devices, which have been widely discussed in the past few years due to this issue. Moving forward, it’s notable that a large amount of our cyberspace is built without security in mind and organisation may not be fully aware of the risks of the technologies they are using. Thus, a valuable method of developing prevention against cyber terrorist threats before they could happen, is by implementing security as one of the integral parts of development in softwares and devices. \\
\par
Whilst our government has many deterrents for those committing cyber attacks, the probability of getting caught is often in the minds of criminals, it could be said that for terrorist and terrorist organisations this is not a concern. As a result, when discussing prevention methods against cyber terrorism as appose to that of cyber crime, the methods must be considered differently due to the perspective of the attacker. Often terrorist have no legislation to follow and are not concerned with the consequences of identification before or during an attack. Concluding that it is of vital importance for preliminary reconnaissance, defence and action to identify attackers is made in the swiftest of nature. \\
\par
Intrusion detection is one of the most active areas of research within cyber terrorism over the past 20 years. Creating safe barriers, both within our systems and physically, are necessary in order to identify occurring attacks in order to implement the right method of mitigation. Many of these techniques, as previously discussed, include the likes of encryption. Passwords, could be seen as one of the oldest methods of intrusion detection. As these methods get more widely used, vulnerabilities become more common. It’s notable that in order to develop the mitigation of an attack, we must constantly develop new intrusion detection systems in order to be as effective as possible during an attack. Not only does this improve the mitigation, but it also allows for a compartmentalisation of a system to allow for a limitation of possible damage and thus protecting valued assets before irreparable damage occurs. Additionally, responses to cyber attacks can be improved by focusing more attention on preserving data during an attack. As discussed by many security professionals across the globe, often data breaches are not recently backed up, thus it is vital to have up to date versions of systems or databases at all times. Limiting the amount of damage caused after a cyber attack is an essential part of incident management. This is a primary stage of recovering and responding to an act of cyber terrorism and enables for future protections.\\

\large
\textbf{7. Write short notes on: }
\normalsize
\begin{center}
	\textbf{Contributions of Mark Zuckerberg}
\end{center}

Mark Elliot Zuckerberg is an American technology entrepreneur and a philanthropist. He is one of the most influential people in this century and also the youngest self made billionaire. He is known for being the co-founder of the multinational social-networking company Facebook.\\
\par
Mark Zuckerberg is best known for co-founding and creating Facebook, which he developed at Harvard University with three classmates: Divya Narendra, Cameron Winklevoss, and Tyler Winklevoss. Prior to Facebook’s creation, Zuckerberg created CourseMatch, a program developed to help students choose their classes based on selections and reviews by other classmates. He went on to develop Facemash, which was the direct predecessor of Facebook; it used photos of classmates and enabled students to vote on which classmate was more attractive. Zuckerberg then began the prototype to Facebook, which was run out of his dorm room until the summer of 2004. After his sophomore year, Zuckerberg dropped out of college and moved his company to Palo Alto, California, and by the end of the year, Facebook had reached the milestone of one million users. TechCrunch reported in June 2017 that the platform now has two billion monthly users.\\
\par
Apart from the social platform Facebook, Facebook, Inc. has developed several open source frameworks and platforms like React Native. There are millions of developers who use the open source libraries provided by Facebook.\\
\par
Zuckerberg has been actively participated in charity and donations. His charitable donations are based on his pledge to donate 99 percent of his wealth to charities over the course of his life. He vowed himself to this by signing the Giving Pledge. The Giving Pledge is a commitment by the world’s wealthiest individuals to dedicate the majority of their wealth into charities and for good causes. Within this pledge, Zuckerberg also wants to promote equality and advance the human potential for future generations. In September 2016, Mark and his wife Priscilla Chan sold \$95 million in Facebook stocks so that they could give the proceeds to charity. They plan to sell the stocks in the amount to at least \$1 billion each year. So far, Zuckerberg has donated \$45 billion to charitable purposes. He and his wife firstly pledged to give donations to charity by writing an open letter to their daughter who was born in late 2015. The birth of their daughter inspired them to change the world and environment and make it a better place for their daughter to live in.\\

\begin{center}
	\textbf{Use of e-government systems in Nepal}\\
\end{center}

Electronic governance or e-governance is the application of information and communication technology (ICT) for delivering government services, exchange of information, communication transactions, integration of various stand-alone systems and services between government-to-citizen (G2C), government-to-business (G2B), government-to-government (G2G), government-to-employees (G2E) as well as back-office processes and interactions within the entire government framework.\\
\par
Nepal is in the phase of shifting towards the system of e-governance from the traditional file based system. Majority of the government services like birth registration, citizenship registration, immigration, taxation, banking etc happen with the use of computers and internet. For example, to find someone's name in the voter's list required the staffs to search for hours in a big and chunky file folder in the past, which can be searched in less than a second using the computerized database. The same ease has occurred in various other fields of government services.\\
\par
Majority of the systems installed in the government offices till now are either based on single node offline setup, interconnection of a set of local nodes via LAN connection. A better alternative to this would be cloud computing. Not only does it reduce the cost and complexity of computing, this also ensures high availability, integrity of data and fault tolerance. So in my opinion, the e-governance system of Nepal should shift towards cloud computing.

\break

\large
\begin{center}
	\underline{\textbf{2018 Fall}}
\end{center}

\textbf{1. a) State major contributions of Jon von Neumann in the field of computing.}\\
\normalsize

John von Neumann was a Hungarian-American mathematician, physicist, computer scientist and polymath. He made major contributions to a number of fields, including mathematics (foundations of mathematics, functional analysis, ergodic theory, representation theory, operator algebras, geometry, topology, and numerical analysis), physics (quantum mechanics, hydrodynamics, and quantum statistical mechanics), economics (game theory), computing (Von Neumann architecture, linear programming, self-replicating machines, stochastic computing), and statistics.\\
\par
Some of his contributions in the field of computing are listed below.

\begin{itemize}

	\item While consulting for the Moore School of Electrical Engineering at the University of Pennsylvania on the EDVAC project, von Neumann wrote an incomplete First Draft of a Report on the EDVAC. This paper described a computer architecture in which the data and the program are both stored in the computer's memory in the same address space. This architecture, commonly known as the \textbf{von Neumann architecture}, is the basis of most modern computer designs.
	\item Von Neumann designed the \textbf{IAS machine} which was the foundation for the IBM to release its commercially successful IBM 704.
	\item He invented the \textbf{merge sort algorithm}, in which the first and second halves of an array are each sorted recursively and then merged.
	\item He also worked on the \textbf{philosophy of artificial intelligence} with Alan Turing when Turing visited Princeton in the 1930s.
	\item He developed the \textbf{Monte Carlo method}, which allowed solutions to complicated problems to be approxiamated by using random numbers.
	\item He developed the \textbf{middle-square method} of generating pseudo-random numbers.
	\item \textbf{Stochastic} computing was first introduced in a pioneering paper by von Neumann in 1953.
	\item Von Neumann created the field of \textbf{cellular automata} without the aid of computers, constructing the first self-replicating automata with pencil and graph paper.
	
\end{itemize}

\large
\textbf{2. a) Define descriptive and normative claims. Explain utilitarianism and deontological theories.}\\
\normalsize
Descriptive ethics is the study of how people do behave, and how they think they should behave. Normative ethics is the study of how people ought to behave. It is an argumentative discipline aimed at sorting out what behaviors (or rules for behavior) would be best.\\
\par
Utilitarianism is the principle that the correct form of action be taken to benefit the greatest number of people. Deontology is defined as the area of ethics involving the responsibility, moral duty and commitment. Both utilitarianism and deontology deal with the ethics and consequences of one’s actions and behavior despite the outcome. a utilitarian approach to morality implies that no moral act (e.g., an act of stealing) or rule (e.g., “Keep your promises”) is intrinsically right or wrong. \\
\par
Rather, the rightness or wrongness of an act or rule is solely a matter of the overall nonmoral good (e.g., pleasure, happiness, health, knowledge, or satisfaction of individual desire) produced in the consequences of doing that act or following that rule. In sum, according to utilitarianism, morality is a matter of the no moral good produced that results from moral actions and rules, and moral duty is instrumental, not intrinsic. Morality is a means to some other end; it is in no way an end in itself. Deontological ethics has at least three important features. First, duty should be done for duty’s sake. The rightness or wrongness of an act or rule is, at least in part, a matter of the intrinsic moral features of that kind of act or rule. \\
\par
For example, acts of lying, promise breaking, or murder are intrinsically wrong and we have a duty not to do these things. This does not mean that consequences of acts are not relevant for assessing those acts. For example, a doctor may have a duty to benefit a patient, and he or she may need to know what medical consequences would result from various treatments in order to determine what would and would not benefit the patient. But consequences are not what make the act right, as is the case with utilitarianism. \\

\large
\textbf{2. b) Define the term whistle blowing. What are the advantages and disadvantages of whistle blowing? Explain steps to solve whistle blowing in an organization.}\\
\normalsize
Whistleblowing is the act of drawing public attention, or the attention of an authority figure, to perceived wrongdoing, misconduct, unethical activity within public, private or third-sector organizations. \\
\par
Some of the advantage of whistleblowing are:

\begin{itemize}
	\item When corporations and government agencies step over legal and ethical lines, whistle-blowers can make these practices public knowledge, which can lead to violators being held accountable.
	\item The federal government's Whistleblower Protection Program protects employees who report violations of various workplace safety, environmental, financial reform and securities laws.
	\item Helps to expose unethical behaviors.
\end{itemize}

Some of the disadvantage of whistleblowing are:
\begin{itemize}
	\item The attention that a whistle-blower case brings, both to the employee and the company, can have a downside. 
	\item Although whistle-blowers may understand that their revelations serve the greater good, they also often endure personal problems from their actions. 
	\item An employee who brings a whistleblowing claim or otherwise provides information to the government can face retaliation from an employer and may have difficulty in getting hired in related fields going forward. 
	\item It diminishes trust in workspace as well.

\end{itemize}

Some of the steps to solve whistle blowing are:

\begin{itemize}
	\item Keep the information strictly confidential; the information should not come from public sources.
	\item Keep this information strictly confidential; the information should not come from public sources.   
	\item If the government decides to bring a case, the whistleblower may be asked to testify at trial or a grand jury proceeding.\\

\end{itemize}


\large
\textbf{3. a) Define values in design. Explain accuracy and democracy in Internet.}\\
\normalsize
Value provides a set of rules and guidelines for designing a system with a certain value in mind.\\
\par
Democracy in internet or Internet democracy incorporates 21st-century information and communications technology to promote democracy. It is a form of government in which all adult citizens are presumed to be eligible to participate equally in the proposal, development, and creation of laws.  E-democracy encompasses social, economic and cultural conditions that enable the free and equal practice of political self-determination. According to Sharique Hassan Manazir, Digital Inclusion  is an inherent necessity of E-Democracy/Digital Democracy just like Social Inclusion is the need of Democracy. \\
\par
The spread of free information through the internet has encouraged freedom and human development. The internet is used for promoting human rights including free speech, religion, expression, peaceful assembly, government accountability, and the right of knowledge and understanding that support democracy. An E-democracy process has been recently proposed in a scientific article for solving a question that has crucial importance for all humans in the 21st century: Democracy in America has become reliant on the Internet because the Internet is a primary source of information for most Americans. The Internet educates people on democracy, helping people stay up to date with what is happening in their government.\\
\par
E-democracy offers greater electronic community access to political processes and policy choices. E-democracy development is connected to complex internal factors, such as political norms and citizen pressures" and in general to the particular model of democracy implemented. E-democracy is therefore highly influenced by both internal factors to a country and by the external factors of standard innovation and diffusion theory.\\

\large
\textbf{3. b) What is personal privacy? How have computers affected privacy?}\\
\normalsize
Personal privacy refers to a person's right to control access to his or her personal information.\\
\par
Nowadays, computers have become a means to share the information. However, they are threat as well. They can harm our privacy in many ways:
\begin{itemize}
	\item Computers are commonly used to communicate personal information, such as conversations between individuals, banking information, usernames and passwords and other sensitive data. Computer usage can also open the door to identity theft and fraud.
	\item If you connect your personal computer to an open Wi-Fi network, you take the chance of exposing personal information to the public because of unsecured wireless networks.
	\item Viruses, malware and spyware programs can make their way onto your computer without your knowledge when you download email attachments, browse unsafe websites or install untrusted software.
	\item People can also send various spams and attractive advertisements in order to mirror ones computer to steal data.
	\item Sometimes we might signup in unknown page that corrupts our computer.\\
\end{itemize}

\large
\textbf{4. a) What do you mean by denial of service? Explain different types of malicious programs.}
\normalsize
Denial-of-service attack (DoS attack) is a cyber-attack in which the perpetrator seeks to make a machine or network resource unavailable to its intended users by temporarily or indefinitely disrupting services of a host connected to the Internet\\
\par
Malicious software, commonly known as malware, is any software that brings harm to a computer system. Different types of malicious software can be enlisted as:

\begin{itemize}
	\item \textbf{Malware}: Regroups viruses, spyware, Trojans, and all sorts of small programs designed to harm your system, steal information, track your activities…etc
	\item \textbf{Spyware}: Spyware (spy software or spyware) is a program designed to collect personal data about users of the infected system and to send them to a third party via the Internet or computer network without permission users.
	\item \textbf{Viruses}: A virus is a piece of malicious computer program designed to replicate itself. This ability to replicate, can affect your computer without your permission and without your knowledge. In layman’s term a classical virus will attach itself to a executable program and systematically replicate to all executable that you run.
	\item \textbf{Worms}: A worm (or worm) is a particular type of virus that can replicate through terminals connected to a network, then to perform certain actions which would impair the integrity of operating systems. 
	\item \textbf{Trojans}: A Trojan looks like a valid program. But in reality it contains hidden features, through which the security mechanisms of the system are bypassed, allowing access to your files (to view, modify or destroy them). Unlike a worm, the Trojan does not replicate: it may stay harmless, in a game or a utility until the scheduled date of its entry into action.
	\item \textbf{Keyloggers}: A keylogger is software that records keystrokes to steal, for example, a password. 
	\item \textbf{Dialer}: The dialers are programs that make up a number to connect your computer to the Internet. It may be safe and legitimate if it is from your ISP for example. However, some dialers are malicious and can move without your knowledge on your machine and dial a number very expensive number.
	\item \textbf{Rootkits}: A rootkit is a very complex malicious code that can merge with your system, and sometimes to very core of the operating system. It is thus able to take full control of a PC without leaving a trace. Detection is difficult, even impossible on some systems.

\end{itemize}


\large
\textbf{4. b) Define terms: Copyright, Trademark, Patent, Trade sheet.}
\normalsize
\begin{itemize}
	\item \textbf{Copyright}: Copyrights protect the expression of ideas. Artistic works are generally considered to be expressions of ideas – books, paintings, songs, movies, and computer programs are examples. Copyright will not protect the process through which a particular work was created or the use of information within it (instructions, etc.). Cookbooks are often used to illustrate the difference between the expression of an idea and the idea itself. Cookbooks cannot be reproduced without permission because they are an expression of ideas (the recipes). However, people can still follow the recipes in the cookbook because they are replicating the ideas contained in the literary work. If the recipes were protected by a patent, users would need permission to follow them, since patents protect particular ideas from being used without authorization.
	\item \textbf{Patent}: A patent is a right, granted by the government, to exclude others from making, using, or selling your invention. Patents protect inventions such as new processes, machines, or chemicals. The central idea is that patents protect ideas, not just expressions of them. The main effect of patents is to give their holders the right to challenge any use of the invention by a third party. He thereby gave a temporary monopoly of exploitation which can be understood as a financial incentive for inventive industrial activities.
	\item \textbf{Trademark}: A trademark (also written trade mark or trade-mark) is a type of intellectual property consisting of a recognizable sign, design, or expression which identifies products or services of a particular source from those of others, although trademarks used to identify services are usually called service marks. The trademark owner can be an individual, business organization, or any legal entity. A trademark may be located on a package, a label, a voucher, or on the product itself. For the sake of corporate identity, trademarks are often displayed on company buildings. It is legally recognized as a type of intellectual property.
	\item \textbf{Trade Secret} :A trade secret is a type of intellectual property in the form of a formula, practice, process, design, instrument, pattern, commercial method, or compilation of information not generally known or reasonably ascertainable by others by which a business can obtain an economic advantage over competitors or customers. In some jurisdictions, such secrets are referred to as confidential information.\\
\end{itemize}

\large
\textbf{5. a) Describe in short IT Policy of Nepal. Do you believe that the Government of Nepal should do more work on the field of cyber law? If yes, give your suggestions}\\
\normalsize

The main achievement of Information and Communication Technology is to provide internet access all over the world. In the present context, when the extension of internet technology has to lead to some criminal acts, in Nepal too “Electronics Transaction Act-2063” was enforced in execution. \\
\par
Government of Nepal should do more work on the field of cyber law. Some suggestions are: 
\begin{itemize}

	\item Although the law is present, it serves little protecting the users online. The law has not been adequately amended as a need of time. So, government should pay full attention to the efficient implementation of law.
	\item Lack of investigation and cyber forensic has diminished the chance of catching the criminals. So, proper investigation should be done.
	\item People should be made award about the cyber crimes so that they can identify the cries easily.

\end{itemize}

\large
\textbf{5. b) Define society. Explain impacts of technlogy in society.}

\normalsize
Society can be defined as a population that occupies the same territory that is subject to the same political authority and that participates in a common culture.\\
\par
The term technology refers to the collection of techniques, skills, methods and processes used in production of goods or services or in the accomplishment of objective, such as scientific investigation. These two terms are largely interrelated together ever since the concept of society was developed during ancient civilizations. These relationships can be discussed briefly under following fields.

\begin{itemize}
\item \textbf{Economic Aspect} : Economy is one of the major foundations for the development of the society. Technology has made industrialization possible, and day by day the cost of production is decreasing gradually by use of newer and efficient technologies. This has strengthened the economy of the society and has upgraded the quality of life of the people. The use of technology have exposed more and more people to employment opportunities, and created many job opportunities in technical field. Technology has made possible electronic payment and electronic commerce system that has made both sellers and buyers easy in selling/buying products. In the other hand, the economy of society directly influences the pace of development of the technology. A society that is economically backwards cannot invest much money in research and production of advanced technologies whereas a society that is economically stable can.

\item \textbf{Socio-Cultural Aspect}  : Technology has drastically raised the awareness level of people in the society. It has made easier for people to access trustworthy and reliable source of information from all around the world. People are able to share and showcase their traditions and cultures via various social media. The increase in social awareness has drastically reduced the number of crimes in the society. Use of advanced technology in field of education has ensured participation of even more people towards quality education. In other hand, the cultures and norms of a society dictates what sort of advancement is acceptable and what is not.

%\begin{figure}[h!] 
%\centering
%	\includegraphics[width=0.5\linewidth]{tech-and-society}
%\end{figure}

\item \textbf{Political Aspect} : Technology has made possible the concept of e-governance. The use of technology in various public sectors have made people easier to receive services provided by the government. It has been easier for political bodies to reach people and put forward their agendas more efficiently by the use of technology. In other hand, the political environment of the society has direct influence on the development of technology, because it is the politics that governs the laws and acts regarding the use of technology.

\item \textbf{Environmental Aspect} : Technology has raised the awareness of people to make effort on the conservation of the environment. Better waste management techniques, pollution control techniques, mass production techniques etc. have been developed, thanks to the advancements in technology. In other hand, technical advancements require a lot of natural resources and these are not possible in an environment where the resources are depleted and scarce.\\

\end{itemize}


\large
\textbf{6. Case Study}\\
\normalsize
\textbf{6. a) Why is access to the Internet important for democracy?}\\
We are living in the Digital Age and in the same way that the internet can change the relationship between governments and citizens for the better. The increased involvement of people in political debate is evident on an even greater scale on social networking sites such as Twitter and Facebook. The internet allows for greater freedom of expression, facilitating citizens' ability to challenge and criticize: a basic democratic right. This digital age is empowering citizens. People, becoming more knowledgeable, can make informed decisions on matters ranging from their family's healthcare to travel. By putting public data online the government is becoming increasingly transparent and so more accountable which again works in the people's favor. The government must be in a position to guarantee where appropriate that online communications are secure and that they do not violate people's privacy. The Internet provides a great platform for political discussion, debate and deliberation that leads to an increase in political participation, which also fosters the formation of social capital, and helps democracy. Government accountability, and have checks and balances in a constitutional system, are crucial elements required for the functioning of a real democracy. Internet and concepts like e-governance, help create a transparency between the government and the citizens, helps understand each other’s problems a little more clearly, and builds and closer relationship between the two. This leads to a more efficient governance, and assists in building trust between the government and the people. The increased political participation, and improved transparency as a result of the internet, has proven to have resulted in rooting out corruption, better monitoring of election results, and hence helping safeguard the democratic principles. Demonstrations, activisms, revolutions, elections, and other processes that may be substantially influenced by digital networking all help are vertical accountability (from the citizens to the government).\\

\large
\textbf{7. Write short notes on:}
\normalsize

\begin{center}
	\textbf{History of Networking}
\end{center}

The history of networking evolved from 1961. And evolution occurred following years as:\\
\textbf{1961}		The idea of ARPANET, one of the earliest computer networks, was proposed by Leonard Kleinrock .
\\ \textbf{1965}		The term "packet" was coined by Donald Davies in 1965, to describe data sent between computers over a network.
\\ \textbf{1969}		ARPANET was one of the first computer networks to use packet switching. 
\\ \textbf{1969}		The first RFC surfaced in April 1969, as a document to define and provide information about computer communications, network protocols, and procedures.
\\ \textbf{1969}		The first network switch and IMP (Interface Message Processor) was sent. It was used to send the first data transmission on ARPANET.
\\ \textbf{1969}		The Internet was officially born, with the first data transmission being sent between UCLA and SRI 
\\ \textbf{1970}		Steve Crocker and a team at UCLA released NCP (NetWare Core Protocol) in 1970. NCP is a file sharing protocol for use with NetWare.
\\ \textbf{1971}		Ray Tomlinson sent the first e-mail in 1971.
\\ \textbf{1971}		ALOHAnet, a UHF wireless packet network, is used in Hawaii to connect the islands together. 
\\ \textbf{1973}		Ethernet is developed by Robert Metcalfe in 1973 while working at Xerox PARC.
\\ \textbf{1973}		The first international network connection, called SATNET, is deployed in 1973 by ARPA.
\\ \textbf{1973}		An experimental VoIP call was made .
\\ \textbf{1974} 	The first routers were used at Xerox in 1974. However, these first routers were not considered true IP routers.
\\ \textbf{1976}		Ginny Strazisar developed the first true IP router, originally called a gateway, in 1976.
\\ \textbf{1978}		Bob Kahn invented the TCP/IP protocol for networks and developed it, with help from Vint Cerf, in 1978.
\\ \textbf{1981}		Internet protocol version 4, or IPv4, was officially .
\\ \textbf{1981}		BITNET was created in 1981 as a network between IBM mainframe systems in the United States.
\\ \textbf{1981}		CSNET (Computer Science Network) was developed by the U.S. National Science Foundation in 
\\ \textbf{1983}		ARPANET finished the transition to using TCP/IP in 1983.
\\ \textbf{1983}		Paul Mockapetris and Jon Postel implement the first DNS in 1983.
\\ \textbf{1986}		The NSFNET (National Science Foundation Network) came online .
\\ \textbf{1986}		BITNET II was created in 1986 to address bandwidth issues with the original BITNET.
\\ \textbf{1988}		The first T1 backbone was added to ARPANET in 1988.
\\ \textbf{1988}		WaveLAN network technology, the official precursor to Wi-Fi, was introduced to the market by AT\&T, Lucent, and NCR in 1988.
\\ \textbf{1988}		Details about network firewall technology was first published in 1988. 
\\ \textbf{1990}		Kalpana, a U.S. network hardware company, developed and introduced the first network switch in 1990.
\\ \textbf{1996}		IPv6 was introduced in 1996 as an improvement over IPv4.
\\ \textbf{1997}	T	he first version of the 802.11 standard for Wi-Fi was introduced .
\\ \textbf{1999}		The 802.11a standard for Wi-Fi was made official in 1999, designed to use the 5 GHz band and provide transmission speeds up to 25 Mbps.
\\ \textbf{1999}		802.11b devices were available to the public starting mid-1999, providing transmission speeds up to 11 Mbps.
\\ \textbf{1999}		The WEP encryption protocol for Wi-Fi is introduced in September 1999, for use with 802.11b.
\\ \textbf{2003}		802.11g devices were available to the public starting in January 2003, providing transmission speeds up to 20 Mbps.
\\ \textbf{2003}		The WPA encryption protocol for Wi-Fi is introduced in 2003, for use with 802.11g.
\\ \textbf{2003}		The WPA2 encryption protocol is introduced in 2004, as an improvement over and replacement for WPA. All Wi-Fi devices are required to be WPA2 certified by 2006.
\\ \textbf{2009}	The 802.11n standard for Wi-Fi was made official in 2009. It provides higher transfer speeds over 802.11a and 802.11g, and it can operate on the 2.4 GHz and 5 GHz bandwidths.
\\ \textbf{2018}	The Wi-Fi Alliance introduced WPA3 encryption for Wi-Fi in January 2018, which includes security enhancements over WPA2.



\large
\begin{center}
	\textbf{\underline{2017 Fall}}\\
\end{center}

\large
\textbf{1. b) How can IT facilitate health and education sector? Put your ideas.}\\
\normalsize
IT in health sector supports health information management across computerized systems and the secure exchange of health information between consumers, providers, payers, and quality monitors. Some of the impacts of IT in health sector are:

\begin{itemize}
	\item Improve health care quality or effectiveness:
	\item Increase health care productivity or efficiency;
	\item Prevent medical errors and increase health care accuracy and procedural correctness;
	\item Reduce health care costs;
	\item Increase administrative efficiencies and healthcare work processes;
	\item Decrease paperwork and unproductive or idle work time;
	\item Extend real-time communications of health informatics among health care professionals; 
	\item Expand access to affordable care.
	\item Early detection of infectious disease outbreaks around the country;
	\item Improved tracking of chronic disease management;
	\item Evaluation of health care based on value enabled by the collection of de-identified price and quality information that can be compared.

\end{itemize}

In education sector:
\begin{itemize}
	\item Information technology makes it easy to access academic information at any time. Both students and teachers use Information technology to acquire and exchange educational material. 
	\item Students can easily access academic data using computers and new technologies like mobile phone application.
	\item Information technology has helped students learn in groups and it has also helped teachers teach students in groups.
	\item Information technology has changed the way we learn and interpret information. The use of audio-visual education helps students learn faster and easily.
	\item Information technology enables students across the globe to study from anywhere through online education.
\end{itemize}


\large
\textbf{2. a) Government has issued Secure Password Practices. Describe its gist. Also put some light on objectives of IT policies of Nepal.}\\
\normalsize
Government has issued Secure Password Practices so that cyber crimes will be eliminated. Most of the people while using internets keep the password simple in order to remember but weak passwords are likely to be attacked soon. So some of the practices are:
\begin{itemize}
	\item Adopt Long Passphrases. 
	\item Avoid Periodic Changes. 
	\item Create Password Blacklist. 
	\item Implement Two-Factor Authentication. 
	\item Add Advanced Authentication Methods. 
	\item Apply Password Encryption. 
	\item Protect Accounts of Privileged Users. 
	\item Ensure Secure Connection.
\end{itemize}

Some of the objectives of IT policy of Nepal are:
\begin{itemize}
	\item Work on theoretical and research based areas related to Information Technology.
	\item Provide training, tutorials and related technical works for the encouragement of IT literacy among different groups of people.
	\item Make a keen participation in seminars, research discussion and workshops so as to provide a broader and greater view for IT policy.
	\item To actively pursue burning ICT issues of the country, this requires professional and academic attentions.
	\item To become a professional forum for the IT students of the whole nation.
	\item To advocate and act for the general interests of the IT students of the country.
	\item To collect and disseminate information regarding the ICT sector of the country.
	\item To be aware of the legal and regulatory issues affecting the ICT industry and to actively support the progressive legislation that clarifies these issues without restricting its growth or creativity.
	\item To explore new horizon of challenges and opportunities that can boost the technological, economic and educational scenario of the country.
\end{itemize}

\large
\textbf{2. b) What are trademarks and trade secrets? How has GoN protected them? Explain in brief.}\\
\normalsize
A trademark is a type of intellectual property consisting of a recognizable sign, design, or expression which identifies products or services of a particular source from those of others, although trademarks used to identify services are usually called service marks. A trade secret is a type of intellectual property in the form of a formula, practice, process, design, instrument, pattern, commercial method, or compilation of information not generally known or reasonably ascertainable by others by which a business can obtain an economic advantage over competitors or customers.\\
\par
Federal trademark registration is an important protection that can go a long way to ensuring that you reap the rewards of your hard work and investments by defending you from those who would infringe on your brand. Too often, securing a federal trademark registration can be a daunting task. The firm provides a full suite of services to help you determine how to best secure your trademark rights and navigate the often-confusing legal waters of trademark registration. If you are considering starting a new business or have been running your business for years—no matter if you are working out of your garage or have established yourself in the market—federal trademark registration is an important part of any business. \\
\par
Nearly every business has proprietary information, but trade secret protection is difficult to maintain and easy to lose. Developing procedures for handling confidential information can range from day-to-day policies, employment agreements, licensing agreements, and non-disclosures which may be necessary when seeking investors. However, when trade secrets get out they are gone forever. If your business has proprietary information, techniques, or products that qualify for protection, the Law Office of Ken Brady will help you protect your trade secret.\\

\large
\textbf{3. a) Describe the evolution of high level programming languages.}
\normalsize
The lack of portability between different computers led to the development of high-level languages. Further, it was recognized that the closer the syntax, rules, and mnemonics of the programming language could be to  natural language  the less likely it became that the programmer would inadvertently introduce errors (called  bugs ) into the program. Hence, in the mid-1950s a third generation of languages came into use. These algorithmic, or procedural, languages are designed for solving a particular type of problem. Unlike machine or symbolic languages, they vary little between computers. They must be translated into machine code by a program called a compiler or interpreter.\\
\par
Early computers were used almost exclusively by scientists, and the first high-level language, Fortran was developed (1953–57) for scientific and engineering applications by John Backus at the IBM Corp. A program that handled recursive algorithms better, LISP was developed by John McCarthy at the Massachusetts Institute of Technology in the early 1950s; implemented in 1959, it has become the standard language for the artificial intelligence community. COBOL the first language intended for commercial applications, is still widely used; it was developed by a committee of computer manufacturers and users under the leadership of Grace Hopper, a U.S. Navy programmer, in 1959. ALGOL developed in Europe about 1958, is used primarily in mathematics and science, as is APL published in the United States in 1962 by Kenneth Iverson. PL/1 developed in the late 1960s by the IBM Corp., and ADA developed in 1981 by the U.S. Dept. of Defense, are designed for both business and scientific use.\\
\par
BASIC was developed by two Dartmouth College professors, John Kemeny and Thomas Kurtz, as a teaching tool for undergraduates (1966); it subsequently became the primary language of the personal computer revolution. In 1971, Swiss professor Nicholas Wirth developed a more structured language for teaching that he named Pascal (for French mathematician Blaise Pascal , who built the first successful mechanical calculator). Modula 2, a Pascallike language for commercial and mathematical applications, was introduced by Wirth in 1982. Ten years before that, to implement the UNIX operating system, Dennis Ritchie of Bell Laboratories produced a language that he called C; along with its extensions, called C++, developed by Bjarne Stroustrup of Bell Laboratories, it has perhaps become the most widely used general-purpose language among professional programmers because of its ability to deal with the rigors of object-oriented programming . Java is an object-oriented language similar to C++ but simplified to eliminate features that are prone to programming errors. Java was developed specifically as a network-oriented language, for writing programs that can be safely downloaded through the Internet and immediately run without fear of computer viruses. Using small Java programs called applets, World Wide Web pages can be developed that include a full range of multimedia functions.\\
\par
Fourth-generation languages are nonprocedural they specify what is to be accomplished without describing how. The first one, FORTH, developed in 1970 by American astronomer Charles Moore, is used in scientific and industrial control applications. Most fourth-generation languages are written for specific purposes. Fifth-generation languages, which are still in their infancy, are an outgrowth of artificial intelligence research. PROLOG developed by French computer scientist Alain Colmerauer and logician Philippe Roussel in the early 1970s, is useful for programming logical processes and making deductions automatically.\\
\par
Many other languages have been designed to meet specialized needs. GPSS is used for modeling physical and environmental events, and SNOBOL is designed for pattern matching and list processing. LOGO, a version of LISP, was developed in the 1960s to help children learn about computers. PILOT is used in writing instructional software, and Occam is a no sequential language that optimizes the execution of a program's instructions in parallel-processing systems. There are also procedural languages that operate solely within a larger program to customize it to a user's particular needs. \\

\large
\textbf{ 3. b) Cyber Terrorism has been a major issue in today's context. What do you think how to deal with this issue? Put your views.}\\
\normalsize
Cyberterrorism is the use of the Internet to conduct violent acts that result in, or threaten, loss of life or significant bodily harm, in order to achieve political or ideological gains through threat or intimidation. It is also sometimes considered an act of Internet terrorism where terrorist activities, including acts of deliberate, large-scale disruption of computer networks, especially of personal computers attached to the Internet by means of tools such as computer viruses, computer worms, phishing, and other malicious software and hardware methods and programming scripts.\\
\par
Today, where every people in the world have easy and unrestricted access to the internet, the crimes like cyber terrorism has increased significantly.

Over recent years vulnerabilities in softwares and new technologies have proved that security is not often at the forefront of priority during it’s development. An example being, The Internet of Things (IoT) devices, which have been widely discussed in the past few years due to this issue. Moving forward, it’s notable that a large amount of our cyberspace is built without security in mind and organisation may not be fully aware of the risks of the technologies they are using. Thus, a valuable method of developing prevention against cyber terrorist threats before they could happen, is by implementing security as one of the integral parts of development in softwares and devices. \\
\par
Whilst our government has many deterrents for those committing cyber attacks, the probability of getting caught is often in the minds of criminals, it could be said that for terrorist and terrorist organisations this is not a concern. As a result, when discussing prevention methods against cyber terrorism as appose to that of cyber crime, the methods must be considered differently due to the perspective of the attacker. Often terrorist have no legislation to follow and are not concerned with the consequences of identification before or during an attack. Concluding that it is of vital importance for preliminary reconnaissance, defence and action to identify attackers is made in the swiftest of nature. \\
\par
Intrusion detection is one of the most active areas of research within cyber terrorism over the past 20 years. Creating safe barriers, both within our systems and physically, are necessary in order to identify occurring attacks in order to implement the right method of mitigation. Many of these techniques, as previously discussed, include the likes of encryption. Passwords, could be seen as one of the oldest methods of intrusion detection. As these methods get more widely used, vulnerabilities become more common. It’s notable that in order to develop the mitigation of an attack, we must constantly develop new intrusion detection systems in order to be as effective as possible during an attack. Not only does this improve the mitigation, but it also allows for a compartmentalisation of a system to allow for a limitation of possible damage and thus protecting valued assets before irreparable damage occurs. Additionally, responses to cyber attacks can be improved by focusing more attention on preserving data during an attack. As discussed by many security professionals across the globe, often data breaches are not recently backed up, thus it is vital to have up to date versions of systems or databases at all times. Limiting the amount of damage caused after a cyber attack is an essential part of incident management. This is a primary stage of recovering and responding to an act of cyber terrorism and enables for future protections.\\


\large
\textbf{4. Case Study}\\
\normalsize
\textbf{4. a) Taking the reference of the given case, with formation of new government in US, the ethical relativism can be clearly perceived. Describe what ethical relatism is, in this context.}\\
Ethical relativism is the theory that holds that morality is relative to the norms of one's culture. That is, whether an action is right or wrong depends on the moral norms of the society in which it is practiced. The same action may be morally right in one society but be morally wrong in another. For the ethical relativist, there are no universal moral standards, standards that can be universally applied to all peoples at all times. The only moral standards against which a society's practices can be judged are its own. If ethical relativism is correct, there can be no common framework for resolving moral disputes or for reaching agreement on ethical matters among members of different societies. Most ethicists reject the theory of ethical relativism. Some claim that while the moral practices of societies may differ, the fundamental moral principles underlying these practices do not. For example, in some societies, killing one's parents after they reached a certain age was common practice, stemming from the belief that people were better off in the afterlife if they entered it while still physically active and vigorous. While such a practice would be condemned in our society, we would agree with these societies on the underlying moral principle -- the duty to care for parents. Societies, then, may differ in their application of fundamental moral principles but agree on the principles.\\


\textbf{4. c) What should be the role of professionals in dealing with the aforementioned condition?}\\
The role of professionals in dealing with above mentioned condition are:
\begin{itemize}
	\item Set and implement user access controls and identity and access management systems
	\item Monitor network and application performance to identify and irregular activity
	\item Perform regular audits to ensure security practices are compliant
	\item Deploy endpoint detection and prevention tools to thwart malicious hacks
	\item Set up patch management systems to update applications automatically
	\item Implement comprehensive vulnerability management systems across all assets on-premises and in the cloud
	\item Work with IT operations to set up a shared disaster recovery/business continuity plan
	\item Work with HR and/or team leads to educate employees on how to identify suspicious activity.
\end{itemize}

\large
\textbf{5. a) What is conflict of interest? Describe its types.}\\
\normalsize
A conflict of interest is a situation in which a person or organization is involved in multiple interests, financial or otherwise, and serving one interest could involve working against another. Some of the types of it are:\\

\textbf{Actual conflict of interest}\\
There is a real conflict between an employee or director’s public duties and private interests.\\
\par
\textbf{Potential conflict of interest}\\
An employee or director has private interests that could conflict with their public duties. This refers to circumstances where it is foreseeable that a conflict may arise in future and steps should be taken now to mitigate that future risk.\\
\par
\textbf{Perceived conflict of interest}\\
The public or a third party could form the view that an employee or director’s private interests could improperly influence their decisions or actions, now or in the future.\\
\par
\textbf{Conflict of duty}\\
Will arise when a person is required to fulfill two or more roles that may actually, potentially or be perceived to be in conflict with each other.\\

\large
\textbf{5. b) What is netiquette? How should we behave while using emails? Explain.}\\
\normalsize
Netiquette refers to Internet etiquette. This simply means the use of good manners in online communication such as e-mail, forums, blogs, and social networking sites to name a few. Some of the ways to behave while using email are:
\begin{itemize}
	\item One of the most important things to consider when it comes to e-mail etiquette is whether the matter you're discussing is a public one, or something that should be talked about behind closed doors.
	\item Do not assume the person receiving your e-mail knows who you are, or remembers meeting you.
	\item E-mailing with bad news, firing a client or vendor, expressing anger, reprimanding someone, disparaging other people in e-mails are disturbing.
	\item Refrain from discussing confidential information in e-mails such as someone's tax information or the particulars of a highly-sensitive business deal. 
	\item Never open an old e-mail, hit Reply, and send a message that has nothing to do with the previous one. 
\end{itemize}

\large
\textbf{5. c) "Accuracy vs Democracy" has been a raising concern in the internet. Which faction would you prefer? Put your views.}\\
\normalsize
Democracy in internet or Internet democracy incorporates 21st-century information and communications technology to promote democracy. It is a form of government in which all adult citizens are presumed to be eligible to participate equally in the proposal, development, and creation of laws.  E-democracy encompasses social, economic and cultural conditions that enable the free and equal practice of political self-determination. According to Sharique Hassan Manazir, Digital Inclusion  is an inherent necessity of E-Democracy/Digital Democracy just like Social Inclusion is the need of Democracy. \\
\par
The spread of free information through the internet has encouraged freedom and human development. The internet is used for promoting human rights including free speech, religion, expression, peaceful assembly, government accountability, and the right of knowledge and understanding that support democracy. An E-democracy process has been recently proposed in a scientific article for solving a question that has crucial importance for all humans in the 21st century: Democracy in America has become reliant on the Internet because the Internet is a primary source of information for most Americans. The Internet educates people on democracy, helping people stay up to date with what is happening in their government.\\
\par
E-democracy offers greater electronic community access to political processes and policy choices. E-democracy development is connected to complex internal factors, such as political norms and citizen pressures" and in general to the particular model of democracy implemented. E-democracy is therefore highly influenced by both internal factors to a country and by the external factors of standard innovation and diffusion theory.\\


\large
\textbf{6. a) What is offensive speech and censorship? Explain the importance of censorship illustrating an example.}\\
\normalsize
Offensive speech is a statement intended to demean and brutalize another, or the use of cruel and derogatory language on the basis of real or alleged membership in a social group.\\
\par
Censorship is the suppression of speech, public communication, or other information, on the basis that such material is considered objectionable, harmful, sensitive, or inconvenient.\\
\par
Censorship is important for the following:
\begin{itemize}
	\item At involves restricting materials relating to obscenity, profanity, pornography and indecency.
	\item It also protects the privacy and reputation of an individual.
	\item For supervision and control of the information and ideas circulated within a society.
	\item For the examination of media including books, periodicals, plays, motion pictures, and television and radio programs for the purpose of altering or suppressing parts thought to be offensive.
\end{itemize}

\large
\textbf{6. b) What are malicious programs? Describe different types of such programs.}\\
\normalsize

Malicious programs, commonly known as malware, is any software that brings harm to a computer system. Different types of malicious software can be enlisted as:

\begin{itemize}
	\item \textbf{Malware}: Regroups viruses, spyware, Trojans, and all sorts of small programs designed to harm your system, steal information, track your activities…etc
	\item \textbf{Spyware}: Spyware (spy software or spyware) is a program designed to collect personal data about users of the infected system and to send them to a third party via the Internet or computer network without permission users.
	\item \textbf{Viruses}: A virus is a piece of malicious computer program designed to replicate itself. This ability to replicate, can affect your computer without your permission and without your knowledge. In layman’s term a classical virus will attach itself to a executable program and systematically replicate to all executable that you run.
	\item \textbf{Worms}: A worm (or worm) is a particular type of virus that can replicate through terminals connected to a network, then to perform certain actions which would impair the integrity of operating systems. 
	\item \textbf{Trojans}: A Trojan looks like a valid program. But in reality it contains hidden features, through which the security mechanisms of the system are bypassed, allowing access to your files (to view, modify or destroy them). Unlike a worm, the Trojan does not replicate: it may stay harmless, in a game or a utility until the scheduled date of its entry into action.
	\item \textbf{Keyloggers}: A keylogger is software that records keystrokes to steal, for example, a password. 
	\item \textbf{Dialer}: The dialers are programs that make up a number to connect your computer to the Internet. It may be safe and legitimate if it is from your ISP for example. However, some dialers are malicious and can move without your knowledge on your machine and dial a number very expensive number.
	\item \textbf{Rootkits}: A rootkit is a very complex malicious code that can merge with your system, and sometimes to very core of the operating system. It is thus able to take full control of a PC without leaving a trace. Detection is difficult, even impossible on some systems.

\end{itemize}



\large
\textbf{6. c) What are rights? Does a professional have special rights? Describe in brief.}\\
\normalsize
Rights are the fundamental normative rules about what is allowed of people or owed to people, according to some legal system, social convention, or ethical theory. Professional has their special rights as well. These professional rights include:
\begin{itemize}
	\item Right of Professional Conscience: This is a basic right which explains that the decisions taken while carrying on with the duty, where they are taken in moral and ethical manner, cannot be opposed. The right of professional conscience is the moral right to exercise professional judgment in pursuing professional responsibilities. It requires autonomous moral judgment in trying to uncover the most morally reasonable courses of action, and the correct courses of action are not always obvious.
\item Right of Conscientious Refusal: The right of conscientious refusal is the right to refuse to engage in unethical behavior. This can be done solely because it feels unethical to the doer. This action might bring conflicts within the authority-based relationships. The two main situations to be considered here are:
	\begin{itemize}
		\item When it is already stated that certain act is unethical in a widely shared agreement among all the employees.
		\item When there occurs disagreement among considerable number of people whether the act is unethical.
	\end{itemize}
\item Right to Recognition: A professional has a right to the recognition of one’s work and accomplishments. He also has right to speak about the work one does by maintaining confidentiality and can receive external recognition. The right for internal recognition which includes patents, promotions, raises etc. along with a fair remuneration, is also a part of it. The fulfillment of right to recognition motivates the employee to be a trustful member of the organization, which also benefits the employer. This makes the employee morally bound which enhances the ethical nature to be abided by the professional ethics.

\end{itemize}

\large
\textbf{7. Write short notes on.}\\

\normalsize
\begin{center}
	\textbf{Mark Zuckerberg}
\end{center}

Mark Elliot Zuckerberg is an American technology entrepreneur and a philanthropist. He is one of the most influential people in this century and also the youngest self made billionaire. He is known for being the co-founder of the multinational social-networking company Facebook.\\
\par
Mark Zuckerberg is best known for co-founding and creating Facebook, which he developed at Harvard University with three classmates: Divya Narendra, Cameron Winklevoss, and Tyler Winklevoss. Prior to Facebook’s creation, Zuckerberg created CourseMatch, a program developed to help students choose their classes based on selections and reviews by other classmates. He went on to develop Facemash, which was the direct predecessor of Facebook; it used photos of classmates and enabled students to vote on which classmate was more attractive. Zuckerberg then began the prototype to Facebook, which was run out of his dorm room until the summer of 2004. After his sophomore year, Zuckerberg dropped out of college and moved his company to Palo Alto, California, and by the end of the year, Facebook had reached the milestone of one million users. TechCrunch reported in June 2017 that the platform now has two billion monthly users.\\
\par
Apart from the social platform Facebook, Facebook, Inc. has developed several open source frameworks and platforms like React Native. There are millions of developers who use the open source libraries provided by Facebook.\\
\par
Zuckerberg has been actively participated in charity and donations. His charitable donations are based on his pledge to donate 99 percent of his wealth to charities over the course of his life. He vowed himself to this by signing the Giving Pledge. The Giving Pledge is a commitment by the world’s wealthiest individuals to dedicate the majority of their wealth into charities and for good causes. Within this pledge, Zuckerberg also wants to promote equality and advance the human potential for future generations. In September 2016, Mark and his wife Priscilla Chan sold \$95 million in Facebook stocks so that they could give the proceeds to charity. They plan to sell the stocks in the amount to at least \$1 billion each year. So far, Zuckerberg has donated \$45 billion to charitable purposes. He and his wife firstly pledged to give donations to charity by writing an open letter to their daughter who was born in late 2015. The birth of their daughter inspired them to change the world and environment and make it a better place for their daughter to live in.\\


\begin{center}
	\textbf{Safety in critical system}
\end{center}

Safety-critical systems are those systems whose failure could result in loss of life, significant property damage, or damage to the environment. There are many well known examples in application areas such as medical devices, aircraft flight control, weapons, and nuclear systems. Many modern information systems are becoming safety-critical in a general sense because financial loss and even loss of life can result from their failure. Future safety-critical systems will be more common and more powerful. From a software perspective, developing safety critical systems in the numbers required and with adequate dependability is going to require significant advances in areas such as specification, architecture, verification, and process. The very visible problems that have arisen in the area of information-system security suggests that security is a major challenge also.

To see why safety is so important in critical system, let us see some of the worst disasaters in history of mankind that are caused due to the errors in the critical system:

\begin{itemize}
	\item The Therac-25 radiation machine killed at least six people due to overdose of radiation, all due to the fault in the software installed within the machine.
	\item In February 1991, an American Patriot Missile battery in Saudi Arabia failed to track and intercept an incoming Iraqi Scud missile due to the errors in the calculation of time in the software, and killed 28 soldiers and injured 100 other people.
	\item In 1997, Korean Air Flight 801 crashed just three miles short of the runway in Guam due to a fault with the Ground Proximity Warning System, killing 228 people.
	\item in June 1994, a Royal Air Force Chinook helicopter crashed into the Mull of Kintyre, killing 29. Investigation later revealed that it was caused due to software bug in the aircraft's engine control computer.
\end{itemize}

To minimize the risks of failure in such critical systems, the following safety measures can be practiced.
\begin{itemize}
	\item Design the system to be fail-safe as a fundamental requirement as early as the conception of idea.
	\item Employ aggressive and non-comprising tests to ensure that the system works as it is required to do, without even a simplest error.
	\item Regular maintainance of the system and its software.
	\item Always make available the emergency procedures should anything go wrong.
\end{itemize}

\begin{center}
	\large
	\textbf{\underline{THE-END}}
\end{center}

\end{document}